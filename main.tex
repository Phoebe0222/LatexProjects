\documentclass[12pt]{article}
\usepackage[utf8]{inputenc}
\usepackage[T1]{fontenc}
\usepackage{graphicx}
\usepackage{xcolor}
\usepackage{biblatex}
\addbibresource{example.bib}
\input{defs}

\usepackage{lipsum}

%%%%%%%%%%%%%%%
% Title Page
\title{APC3 Assignment}
\author{Phoebe Liang}
\date{\today}
\summary{\small
Consulting Actuaries Ltd. has prepared this report in accordance with the request of the investment committee of XYZ General Ltd. In accordance with relevant financial services legislation, Consulting Actuaries Ltd. is as Australian Financial Services licence holder and a team of Fellows of the Institute of Actuaries of Australia has permitted sound and legitimate advice given in this report. We understand that XYZ, which writes motor vehicle, home and contents, and building insurance policies, is expanding its policy offerings with electricity blackout insurance (EBI), to be made available to retail and commercial customers in Victoria. We also understand that the committee is seeking advise on how this product would be structured and how it would be received by the customers, as well as using derivative as a tool of risk management.
}
%%%%%%%%%%%%%%%

\begin{document}
\maketitle

\tableofcontents
\clearpage

%%%%%%%%%%%%%%%
\section{Background}
\begin{flushleft}
This section provides a background to the Australian electricity market, explaining how electricity
is generated by providers, sent to the grid and dispatched to customers.
\end{flushleft}
\subsection{Electricity market in Australia}
\begin{flushleft}
The electricity system in Australia begins with acquisition of these primary energy sources: sunlight, wind, water, coal and gas.\parencite{aemc1} Historically, coal-fired power stations have played a dominating role in the electricity generation in Australia, but renewables such as sunlight, wind, water are growing rapidly to form a larger fraction of supply.\parencite{aer1} \par
\begin{figure}[!h]
  \centering
  \begin{minipage}[b]{0.45\textwidth}
    \includegraphics[width=\textwidth]{Registered_capacity.jpg}
    \caption{Registered Capacity}
  \end{minipage}
  \hfill
  \begin{minipage}[b]{0.45\textwidth}
    \includegraphics[width=\textwidth]{supply_output.jpg}
    \caption{Supply Output}
  \end{minipage}
\end{figure}
Then the generators make electricity from these sources and flows into the transmission network, which later transports electricity to the distribution network. The distribution network then transports electricity to residential and commercial buildings for end users.\parencite{aemc1}
\begin{figure}[!h]
  \centering
  \includegraphics[width=\textwidth]{transport.png}
    \caption{Electricity Transport}
    \parencite{aemo1pic}
\end{figure}
\end{flushleft}
\newpage

\begin{flushleft}
The electricity market in Australia is made up of the following 3 sectors:\par
{\textbf{\large Competitive Wholesale Generation Sector:}}\par
The wholesale national electricity market (NEM) is where generators sell electricity to retailers so they can resell to businesses and households. There are 2 ways in trading electricity in NEM: through spot market or the contract market.\parencite{aemc2} The Australian Energy Market Commission (AEMC) overlooks the states and territories participating in NEM by developing and maintaining the Australian National Electricity Rules (NER). The rules are enforced by the Australian Energy Regulator (AER) and the day-to-day management of NEM is performed by the Australian Energy Market Operator (AEMO).\parencite{aemo1} In Victoria, generators include Loy Yang A and B (supply 30\% electricity needs in Victoria), Yallourn W (supply 22\% needs in Victoria and 8\% needs in NEM), and addition capacity is provided by Pelican Point in South Australia, Tamat Valley in Tasmania and Swanbabnk in Queensland. \par
{\textbf{\large Monopoly Network Business:}}\par
The transmission and distribution networks are poles and wires, transformers, switches, monitoring equipment and everything that make up the electricity grid. Given the intensive investment to construct and manage the network, it is cost effective to have a monopoly network service.\parencite{aemc2} In Victoria, the networks are owned by: Ausnet Service, United Energy Distribution, Powercor Australia, Citipower and Jemena.\par
{\textbf{\large Competitive Retail Sector:}}\par
In the retail sector, customers such as businesses and households buy electricity through retailers. Some large consumers such as manufactures may purchase electricity straight from wholesale market, i.e. through a generator rather than through a retailer.\parencite{aemc2} In this sector, electricity consumption by end users is distributed as the following:
\begin{figure}[!h]
  \centering
  \includegraphics[width=\textwidth]{electricity_consumption.PNG}
    \caption{Electricity Transport}
    \parencite{aemo2}
\end{figure}
\end{flushleft}
\newpage

\subsection{Electricity blackout Compensation}
\begin{flushleft}
Under customers compensation scheme, when end users experience power outages beyond set thresholds, they may be entitled to compensations payments from their electricity distributors in their locations, which are known as "Guaranteed Service Level" payments. However, the amount of payments depends on the nature of events, for example the duration of power interruptions, the number of power interruptions. The payment can go up to \$360 a year.\parencite{cc} \par 
\end{flushleft}

\subsection{Electricity Blackouts in Victoria}
\begin{flushleft}
A power outage, or an electricity blackout is a short-term or a long-term loss of electricity power to a particular area in Wikipedia's definition. As stated in Eaton's Blackout Tracker report, in Victoria about 20 power outages are expected in a year, as taking the average of the number of outages over 2016 to 2018. Also, the average duration of an outage is 199 minutes and 9,923 people are affected per outage in 2018.\parencite{Eaton} Electricity blackouts may be caused by many reasons:
\begin{itemize}
 \item extreme weather (e.g. lighting, floods, heatwaves, bushfires)
 \item falling trees contacting power-lines
 \item animals colliding electricity network equipment
 \item vandalism
 \item underground works, car accident, hot air balloon crash
 \item network overload, equipment failure\parencite{eb2}\parencite{eb1}
\end{itemize}
In Victoria, the leading causes are extreme weather, trees and car accident, with many causes unknown though.\parencite{Eaton} Although it is difficult to put a price on the outages leading to loss of business, one could roughly estimate the cost by valuing what the business would have gained if it were not for the power outage. For example, for a car manufacturer, if it can produce 1,200 cars per day and each car is worth \$50,000, then including the recovery period from the outage, the manufacturer would lose \$60 millions due to one outage.\parencite{MGK} 
\end{flushleft}
\newpage
%%%%%%%%%%%%%%%


%%%%%%%%%%%%%%%
\section{Potential Product Structure}
\begin{flushleft}
This section explain what EBI would seek to achieve for XYZ and its customers, describing a potential structure for the product and explaining the likelihood and severity of electricity blackouts in Victoria.
\end{flushleft}
\fullboxbegin
Note this proposal is a guideline of what should be considered in the product design including definition of an eligible claim and the designated coverage. The actual design is more complex and may involve further investigations. \par
\fullboxend

\subsection{Product Structure}
{\textbf{\large Retail Customers:}}\par
\begin{flushleft}
For retail customers, it might be more attractive to customers to put electricity blackout coverage as an add-on item in their existing household policies. When a retail customers sought electricity blackout insurance, then:
\begin{itemize}
    \item a clear definition of power blackout needs to be established, e.g.:
    \begin{itemize}
        \item unplanned power outages such as outages caused by extreme weather and car accidents, planned outages are not covered
        \item outages caused by vandalism by the insured are not covered
    \end{itemize}
    \item a clear formula for the claim amount needs to be established, e.g.:
    \begin{itemize}
        \item percentage cost of replacing spoiled food over a week for the household;
        \item percentage cost of replacing/repairing damaged electronic devices due to the power surge; 
        \item a formula based entirely on the duration of the power blackout, thereby allowing the insured party to determine the level of cover; or
        \item a formula based on the spike in electricity prices throughout the blackout;
        \item any cost that is not directly related to the outages are not covered, e.g. damage to the fridge caused by spoiled food such as broken seals and stains, or theft and robbery when the alarm system is off during the blackout;
        \item any medical cost and death that are not directly related to the outages are not covered, such as when the life-supporting equipment stop functioning during the blackout which leads to worsen condition or death 
    \end{itemize}
\end{itemize}
\end{flushleft}
{\textbf{\large Commercial Customers:}}\par
\begin{flushleft}
If a commercial enterprise sought power blackout insurance from XYZ, again: 
\begin{itemize}
    \item a clear definition of power blackout needs to be established
    \item a clear formula for the claim amount, e.g.:
\begin{itemize}
    \item percentage of earnings lost and costs of recovering earnings;
    \item a formula based entirely on the duration of the power blackout; or
    \item a formula based on the spike in electricity prices throughout the blackout
period.
\end{itemize}
\end{itemize}
\end{flushleft}
\subsection{Likelihood and Severity}

\begin{flushleft}
In Victoria, as mentioned before, around 20 outages are expected in a year and the average duration of an outage is 199 minutes, which adds up to around 3 days of outage in a year. Therefore, it could be expected to have 3 days of power outage a year in Victoria on average, but outage could be more frequent in area that are more prone to extreme weather such as storms and heatwaves. \par 
While the likelihood of an outage could be estimated quite well, the severity of damages due to electricity blackout could vary in a great range depending on the nature of the business and its ability to bounce back from the blackout. For example, for a supermarket, once the frozen food is spoiled, the cost is the replacement of food and loss of business in-between the replacement; for a manufacture, it could be the loss of efficiency in production; for a financial institute, the IT systems are vulnerable to data loss which requires heavy work to restore data from backup, and the delay service will lead to customers dissatisfaction; for a hospital, the outage could damage health of the patients who rely on life-supporting equipment, or it could delay or force-stop surgeries. \par
Therefore, great caution is needed in dealings with the claims amount defined as percentage of loss of earnings because the claims amount could go enormously large, whereas claims amount based on duration of the outage or the spike in electricity prices throughout the outage could be more predictable. However, some organisations would have resilience plan or backup plan in case of outage, for example, a backup generator or battery, planned shut-down instead of unplanned outage. The insurer should therefore look into whether the insured have resilience plan to determine the likelihood of large claims. To effectively remove small claims, the insurer could also put in place deductibles as the households or business affected by the outage could claim compensations from their electricity distributors under the Victorian compensation scheme. Furthermore, the cost should be estimated taking into account of the recovery time as well rather than just the outage duration. \par
\end{flushleft}
\newpage
%%%%%%%%%%%%%%%


%%%%%%%%%%%%%%%
\section{Pricing and Risk}
This section describes how the proposed product could be priced and the risks for the insurer XYZ. 

\subsection{Pricing}

\begin{flushleft}
The premium charged should be sufficient to meet the cost of claims, operating cost as well as a suitable margin allowed for profit.\par \end{flushleft}
\frameboxbegin{Sound Premium Formula}
Premium = Expected Claims Costs + Expense + Profit Margin - Investment Earnings\par
\begin{itemize}
    \item Expected Claims Costs based on rating structure as discussed below
    \item Expenses including underwriting, administration, claims handling and commissions
    \item Profit Margin as decided by the Broad 
    \item Investment earnings depending on the assumed discount rates
\end{itemize}
\frameboxend
\begin{flushleft}
As EBI is relatively new to the market, there might not be data on expected claims costs or rating structure available, but the following considerations are recommended for constructing the rating structure:\par
For claims amount based on percentage of replacing spoiled food and damaged electronic devices and percentage of loss of earnings: \par
\begin{itemize}
    \item {\textbf{Retail Customers:}}
    \begin{itemize}
        \item postcode (to distinguish area with more extreme weather such as heatwaves and storms etc.); 
        \item size of household (to estimate cost of replacement food);
        \item year of production of electronic devices (to see whether there is surge protector); and
        \item whether there is any life-supporting equipment, and whether there are generators for it. 
    \end{itemize}
    \item {\textbf{Commercial Customers:}}
        \begin{itemize}
        \item postcode; 
        \item size of business; 
        \item number of employees;
        \item profit/production per day;
        \item whether there is backup battery, recovery plan during a outage; and
        \item how often data are backup.
    \end{itemize}
\end{itemize}
For claims amount based entirely on the duration of the power blackout and claims amount based on the spike in electricity prices throughout the blackout, estimated claims costs are not highly related to the factors mentioned above but to the likelihood of outage and price movement of electricity during an outage. Hence, the rating factors can be reduced to postcode and distributor.\par
\end{flushleft}

\subsection{Risk}
\begin{flushleft}
The main risk for the insured is the risk of loss of electricity to the premises being insured, which may include loss of food and convenience for households, and loss of earnings for commercial customers. \par
The risk for XYZ is the unexpectedly higher payments of claims which may lead to loss of profits or even insolvency. This can be caused by higher frequencies of power blackouts and thus higher frequency of claims. Another cause would be the higher claim amount due to:
\begin{itemize}
    \item higher inflation on goods damaged or earnings lost;
    \item longer duration of outage; or
    \item higher price spike throughout the outage for claims.
\end{itemize}
depending on how XYZ define claims in its policy. Since there are 2 ways in trading electricity in NEM, i.e. through spot market or contract market, for claims based by price movement during blackout, XYZ can use derivatives on ASX as an effective risk management tool as discussed in the following sections. 
\end{flushleft}
\newpage
%%%%%%%%%%%%%%%


%%%%%%%%%%%%%%%
\section{Relevant Derivatives}
\begin{flushleft}
This section outlines the relevant derivative products offered on the Australian Stock Exchange (ASX), including how they are quoted and priced, paying particular attention to futures and options in respect of: 
\begin{itemize}
 \item Base - Monthly
 \item Base - Quarterly
 \item Peak - Quarterly
 \item Base - Caps
 \item Base - Strip
\end{itemize}
\end{flushleft}

\subsection{Australian Electricity Futures}
\begin{flushleft}
The underlying commodity of Australian Electricity Futures is electricity energy traded in New South Wales, Victoria, South Australia and Queensland wholesale electricity pool markets conducted by the AEMO. Prices are all quoted in Australian dollar per megawatt hour and has the minimum price movement of \$0.01 per Megawatt hour. 
\end{flushleft}
\subsubsection{Monthly Base Load Futures}
\begin{flushleft}
\textbf{Contract Unit:}\par
1 Megawatt of electrical energy per hour based on a \textbf{base} load profile, where the \textbf{base} load profile is defined as the Wholesale Electricity Pool Market \textbf{base} load period from 00:00 hours Monday to 24:00 hours Sunday over the duration of the \textbf{Contract Month}. The size and tick size of a contract month will vary depending on the number of days and \textbf{base} load hours within the month, for example, 
\frameboxbegin{Size and tick size of a monthly base load future}
A 28 day contract month will equate to 672 Megawatt hours (MWH) and have tick size of \$6.72:\par
   $$ Size = 28 \ days \times 24 \ MWH = 672 \ MWH $$
   $$ Tick\ size = 672 \ MWH \times \$0.01 = \$6.72 $$
\frameboxend
\textbf{Cash Settlement Price:}\par
The Cash Settlement Price is calculated by taking the arithmetic
average of the Wholesale Electricity Pool Market \textbf{base} load spot prices on a half hourly basis over the Contract Month, rounded
to the nearest cent. A Provisional Cash Settlement Price will be
declared on the first Business Day after expiry of the Contract and shall be later confirmed on the third Business Day after expiry.\par
\textbf{Cash Settlement Value:}\par
The Cash Settlement Value is the Cash Settlement Price multiplied by the number of Megawatt hours (MWh) in the underlying Contract
Month.
\end{flushleft}

\subsubsection{Quarterly Base Load Futures}
\begin{flushleft}
\textbf{Contract Unit:}\par
1 Megawatt of electrical energy per hour based on a \textbf{base} load profile, where the \textbf{base} load profile is defined as the Wholesale Electricity Pool Market \textbf{base} load period from 00:00 hours Monday to 24:00 hours Sunday over the duration of the \textbf{Contract Quarter}. The size and tick size of a contract quarter will vary depending on the number of days and base load hours within the quarter, for example, 
\frameboxbegin{Size and tick size of a quarterly base load future}
A 90 day contract month will equate to 2,160 Megawatt hours (MWH) and have tick size of \$21.60:\par
   $$ Size = 90 \ days \times 24 \ MWH = 2,160 \ MWH $$
   $$ Tick\ size = 2,160 \ MWH \times \$0.01 = \$21.60 $$
\frameboxend
\textbf{Cash Settlement Price:}\par
The Cash Settlement Price is calculated by taking the arithmetic
average of the Wholesale Electricity Pool Market \textbf{base} load spot prices on a half hourly basis over the Contract Quarter, rounded to the nearest cent. A Provisional Cash Settlement Price will be declared on the first Business Day after the Last Trading Day of the Contract and shall be later confirmed on the third Business Day after the Last Trading Day\par
\textbf{Cash Settlement Value:}\par
The Cash Settlement Value is the Cash Settlement Price multiplied by the number of Megawatt hours (MWh) in the underlying Contract
Quarter.
\end{flushleft}

\subsubsection{Quarterly Peak Load Futures}
\begin{flushleft}
\textbf{Contract Unit:}\par
1 Megawatt of electrical energy per hour based on a \textbf{peak} load profile, where the \textbf{peak} load profile is defined as the Wholesale Electricity Pool Market \textbf{peak} load period from \textbf{ 07:00am hours to
10:00pm hours Monday to Friday (excluding Public holidays and any other days determined by ASX)} over the duration of the \textbf{Contract Quarter}. The size and tick size of a contract quarter will vary depending on the number of \textbf{peak} days and \textbf{peak} load hours within the quarter, for example, 
\frameboxbegin{Size and tick size of a quarterly peak load future}
A 59 day contract month will equate to 885 Megawatt hours (MWH) and have tick size of \$8.85:\par
    $$ Size = 59 \ days \times 15 \ MWH = 885 \ MWH $$
   $$ Tick\ size = 885 \ MWH \times \$0.01 = \$8.85 $$
\frameboxend
\textbf{Cash Settlement Price:}\par
The Cash Settlement Price is calculated by taking the arithmetic
average of the Wholesale Electricity Pool Market \textbf{peak} load spot prices on a half hourly basis over the Contract Quarter, rounded to the nearest cent. A Provisional Cash Settlement Price will be declared on the first Business Day after the Last Trading Day of the Contract and shall be later confirmed on the third Business Day after the Last Trading Day\par
\textbf{Cash Settlement Value:}\par
The Cash Settlement Value is the Cash Settlement Price multiplied by the number of Megawatt hours (MWh) in the underlying Contract
Quarter.
\end{flushleft}

\subsubsection{Quarterly Base Load \$300 Cap Futures}
\begin{flushleft}
\textbf{Contract Unit:}\par
1 Megawatt of electrical energy per hour based on a \textbf{base} load profile, where the \textbf{base} load profile is defined as the Wholesale Electricity Pool Market \textbf{base} load period from 00:00 hours Monday to 24:00 hours Sunday over the duration of the \textbf{Contract Quarter}. The size and tick size of a contract quarter will vary depending on the number of days and \textbf{base} load hours within the quarter, for example, 
\frameboxbegin{Size and tick size of a quarterly peak load future}
A 59 day contract month will equate to 885 Megawatt hours (MWH) and have tick size of \$8.85:\par
   $$ Size = 90 \ days \times 24 \ MWH = 2,160 \ MWH $$
   $$ Tick\ size = 2,160 \ MWH \times \$0.01 = \$21.60 $$
\frameboxend
\textbf{Cash Settlement Price:}\par
The Cash Settlement Price = $$(C - (300 \times D)) / E,\ where:$$
\begin{itemize}
    \item C = the sum of all base load half hourly spot prices for the Region in the Calendar Quarter greater than \$300.00
    \item D = the total number of base load half hour spot prices for the Region in the Calendar Quarter greater than \$300.00
    \item E = the total number of base load half hour spot prices for the Region in the Calendar Quarter
\end{itemize}
A Provisional Cash Settlement Price will be declared on the first Business Day after the Last Trading Day of the Contract and shall be later confirmed on the third Business Day after the Last Trading Day\par
\textbf{Cash Settlement Value:}\par
The Cash Settlement Value is the Cash Settlement Price multiplied by the number of Megawatt hours (MWh) in the underlying Contract
Quarter.
\end{flushleft}

\subsection{Australian electricity Options}
\begin{flushleft}
There are 3 Australian Electricity Options: calendar year base load strip options, financial year base load strip option, and average rate base load quarterly options. Prices are all quoted in Australian dollar per megawatt hour and has the minimum price movement of \$0.01 per Megawatt hour. 
\end{flushleft}
\subsubsection{Calendar/Financial Year Base Load Strip Options}
\begin{flushleft}
\textbf{Underlying Commodity:}\par
A strip of four Quarterly Base Load Futures Contracts. An equivalent of 1 Megawatt of electrical energy per hour on a base load profile for the respective States (NSW, QLD, VIC and SA) over the duration of a Calendar/Financial Year. \par
\textbf{Exercise Price:} \par
Set at intervals of \$1.00 per MWh. New option exercise prices
created as the underlying futures contract price moves.\par
\textbf{Settlement Method:} \par
\begin{itemize}
    \item All in-the-money options will be automatically exercised.
    \item Unless otherwise directed by a Deny Automatic Exercise Request, all in-the-money Options will be exercised.
    \item Options may be exercised on any business day up to and including the day of expiry. Buyers may exercise in, at and out-of-the-money option positions held, by lodging a notice of manual exercise with ASX Clear (Futures) no later than 12:30pm on the day of expiry. 
\end{itemize}
\textbf{Futures Quarterly Prices:} \par
Upon exercise, the holder will receive four quarterly futures positions at prices equivalent to the option strike price, after applying the current curve ratio determined from the previous business day’s settlement price of the four quarterly futures contracts underlying the relevant Strip Futures Product, as outline below: \par
$$ FP = A \times B/C, \ where $$
\begin{itemize}
    \item FP = Price allocated to each futures contract in the strip resulting from exercised Strip Option
    \item A = The previous day’s Daily Settlement Price for the Contract Quarter for each individual futures contract in the strip
    \item B = Exercise Price
    \item C = Previous Day’s Implied Strip Price, calculated as follows:
\end{itemize}
$$ Implied\ Strip\ Price = F/G $$
\begin{itemize}
    \item F = a + b + c + d
    \begin{itemize}
        \item a = Q1 previous day’s Daily Settlement Price x MWh for Q1
        \item b = Q2 previous day’s Daily Settlement Price x MWh for Q2
        \item c = Q3 previous day’s Daily Settlement Price x MWh for Q3
        \item d = Q4 previous day’s Daily Settlement Price x MWh for Q4
    \end{itemize}
    \item G = Total number of MWh of all four Futures Contracts in the strip
    \item The underlying electricity futures contract with the longest dated expiry will be further adjusted up or down in increments of \$0.01/MWh to the extent that such adjustment will achieve a more accurate implied Exercise Price, as calculated to 4 decimal places.
\end{itemize}
\textbf{Reference Price:} \par
The previous day’s Daily Settlement Price of the underlying Calendar/Financial Year Base Load Strip Futures Product.
\end{flushleft}

\subsubsection{Average Rate Base Load Quarterly Options}
\begin{flushleft}
\textbf{Underlying Commodity:} \par
A Quarterly Base Load Futures Contract. 1 Megawatt of electrical energy per hour on a base load profile for the respective States (NSW, QLD, VIC and SA) over the duration of a Calendar Quarter.\par
\textbf{Exercise Price:} \par
Set at intervals of \$1.00 per MWh. New option exercise prices created as the underlying futures contract price moves.\par
\textbf{Settlement Method:} \par
On the Third Business Day after the Final Trading Day all in-the money options are automatically exercised into the underlying futures contract and cash settled. Deny Automatic Exercise Requests are not permitted. Exercise Requests for out-of-the-money options and/or at-the-money options are not permitted. \par
\textbf{Reference Price:} \par
In accordance with the final Settlement Price of the underlying futures contract as determined on the Third Business Day after the Final Trading Day.
\end{flushleft}
\newpage
%%%%%%%%%%%%%%%

%%%%%%%%%%%%%%%
\section{Liabilities Management}
\begin{flushleft}
This section indicates how derivatives could be used to manage XYZ’s EBI liabilities, including a simple numerical example. \par
Note that XYZ can only use such derivatives strategies for its liabilities management for claims defined base on the spike in electricity prices throughout the blackout period, since claims defined in other ways are not sensitive to the movement in electricity price and it might not be worth the premiums to hedge inflation risk. XYZ can purchase derivatives to gain exposure to electricity price movement and manage its liabilities by effectively stabilising the expected claims amount. Following is a simple numerical example.\par
\end{flushleft}
\frameboxbegin{Example of using derivatives to manage liabilities}
Suppose during an outage in Victoria in 31/12/2018, the average peak price from 7:00am to 10:00pm is \$87.64 per MWH. Suppose the policy defines the claim amount as spike price during the outage, which is around \$87.64. If a customer seeks EBI from XYZ, and he/she satisfies all the conditions to make a claims, then he/she is eligible for a claim amount of \$87.64. Instead of making a sudden payment of \$87.64, XYZ can enter into a contract to sell electricity at the average peak price during the contract quarter (quarterly peak load futures) and a contract buying electricity at the average base price (\$213) during the contract quarter (quarterly base load futures). Then the payment can be reduced to:
$$ payments = \$87.64 + base\ price - peak\ price\ + premiums $$
where, 
\begin{itemize}
    \item average base price = average of (\$91.79+\$97.84+\$99.83+\$92.85) = \$0 \parencite{data1}
    \item average peak price = average of (\$113.57+\$104.23+\$108.40+\$106.94) = \$0 \parencite{data1}
    \item premium = \$77.80 \parencite{data2}
\end{itemize}
Then payment is equal to \$0 which is worse than doing nothing.
\frameboxend
\newpage
%%%%%%%%%%%%%%%

%%%%%%%%%%%%%%%
\section{Residual Risks Management}
\begin{flushleft}
This section gives some examples of residual risks for XYZ and indicates how they could be managed. \par
short-tail, uncertain payments, sudden large payments -> liquidity risks \par
short term payments -> may not get the contracts we want (contract period mismatch, can only get quarterly or monthly contracts, no one is selling/buying, too expensive etc.)\par
\end{flushleft}
%%%%%%%%%%%%%%%

%%%%%%%%%%%%%%%
\section{Conclusion}
\begin{flushleft}
This section gives a conclusion on whether it is worthwhile XYZ pursuing an EBI offering, in whatever form. 

Based on the benefit obtained from hedging, versus the liquidity risks created when
entering futures contracts, we believe it is not worthwhile hedging the exposure to electricity prices in your liabilities. In this case, doing nothing is the best option.
\end{flushleft}
\newpage
%%%%%%%%%%%%%%%

\printbibliography[heading=bibintoc]
\end{document}          
