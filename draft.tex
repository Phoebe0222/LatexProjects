\begin{flushleft}
\begin{verbatim}
https://www.asx.com.au/documents/products/ASX_AustralianElectricityFuturesandOptions_ContractSpecifications_July2015.pdf
https://www.asxenergy.com.au/products/electricity_futures
https://www.aer.gov.au/wholesale-markets/wholesale-statistics/victoria-comparative-base-futures-prices
https://www.aemc.gov.au/energy-system/electricity/electricity-market/spot-and-contract-markets
\end{verbatim}
How the spot market helps keep the lights on:\par
The Australian Energy Market Operator (AEMO) manages the electricity system so power supply and demand is matched simultaneously. The physics of the power system means the electricity supplied by generators must exactly match how much electricity is being used by consumers, or blackouts can happen. The spot market is the mechanism that AEMO uses to match the supply of electricity from power stations with real time consumption by households and businesses. All electricity in the spot market is bought and sold at the spot price. The spot price tells generators how much electricity the market needs at any moment in time to keep the physical power system in balance. When the spot price is increasing, generators  ramp up their output or more expensive generators turn on to sell extra power to the market. For example, a gas peaker or pumped hydro plant may jump in, or a fast-response battery may discharge electricity. When the spot price is decreasing, more expensive generators turn down or off.\par
\begin{table}[!h]
\centering
\caption{Sample table.}
\begin{tabular}{cccc}
\toprule
Value 1 & Value 2 & Value 3 & Value 4\\
\midrule
 odd     & odd   & odd & 1.00 \\
 even    & even  & even& 1.00 \\
 odd     & odd   & odd & 1.00 \\
 even    & even  & even& 1.00 \\
\bottomrule
\end{tabular}
\end{table}
Spot prices reflect how much electricity is being used:\par
Spot prices are currently updated every thirty minutes but this will move to every five minutes in 2021. Prices are usually low in the early hours of the morning, before people wake up and businesses and factories start operating. Spot prices are usually higher in the mid afternoon or evenings, when people and businesses are generally using the most power. There are different spot prices in each of the five NEM regions. Over the year to June 2018, spot prices averaged between \$73 (Queensland) and \$98 (South Australia) per megawatt hour (MWh), equal to 7.3c/kWh and 9.8c/kWh.
\end{flushleft}